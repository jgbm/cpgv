% -*- fill-column: 100 -*-

\documentclass[orivec,envcountsame]{llncs}

\usepackage{amsmath,amssymb,centernot,hyperref,mathwidth,stmaryrd,url,xspace}
\usepackage[table]{xcolor}
\usepackage{colortbl}
\addtolength{\jot}{-3px}

\usepackage{mathpartir}
\mprset{sep=1em}

\usepackage{float}
\floatstyle{boxed}
\restylefloat{figure}

\DeclareMathOperator{\dom}{dom}

%% \newcommand{\Cx}{\mathcal{C}}
%% \newcommand{\Ex}{\mathcal{E}}
%% \newcommand{\Fx}{\mathcal{F}}

\newcommand{\gvtimes}{\times} %\boldsymbol{\otimes}}
\newcommand{\gvplus}{+} %\boldsymbol{\oplus}}

\newcommand{\Cx}{G}
\newcommand{\Ex}{E}
\newcommand{\Fx}{F}

\newcommand{\one}{\mathbf{1}}
\newcommand{\zero}{\mathbf{0}}
\newcommand{\with}{\mathbin\binampersand}
\newcommand{\parr}{\mathbin\bindnasrepma}
\newcommand{\cpdual}[1]{#1^\perp}
\newcommand{\cpbang}[1]{{! #1}}
\newcommand{\cpquery}[1]{{? #1}}
\newcommand{\gvdual}[1]{\overline{#1}}
\newcommand{\channel}[1]{#1^\sharp}
\newcommand{\gvout}[2]{\ensuremath{{!#1.#2}}}
\newcommand{\gvin}[2]{\ensuremath{{?#1.#2}}}
\newcommand{\gvend}{\mkwd{end}}
\newcommand{\lto}{\ensuremath{\multimap}}
\newcommand{\gvforever}[1]{\ensuremath{\Box #1}}
\newcommand{\outterm}{\mkwd{end}_!}
\newcommand{\interm}{\mkwd{end}_?}
\newcommand{\gvserver}[1]{\flat #1}
\newcommand{\gvservice}[1]{\sharp #1}
\newcommand{\trule}[1]{(\textit{#1})}
\newcommand{\cptyp}[2]{#1 \vdash #2}
\newcommand{\gvtyp}[3]{#1 \vdash #2 : #3}

\newcommand{\defiff}{\stackrel{\mathrm{def}}{\iff}}
\newcommand{\defeq}{\triangleq}

\newcommand{\mapname}{\mathsf{map}}
\newcommand{\map}[3]{\mapname^{#1}_{#2}(#3)}
\newcommand{\gvmap}[3]{\mapname^{#1}_{#2}\:#3}

\newcommand{\teval}{\longrightarrow_V}
\newcommand{\ceval}{\longrightarrow}

\newcommand{\rto}{\longrightarrow}
\newcommand{\crto}{\longrightarrow_C}
\newcommand{\ccto}{\longrightarrow_{CC}}

\newcommand{\bigrto}{\Longrightarrow}
\newcommand{\bigcrto}{\Longrightarrow_C}
\newcommand{\bigccto}{\Longrightarrow_{CC}}

\newcommand{\fv}{\ensuremath{\mathsf{fv}}}

\newcommand{\bind}[2]{\ensuremath{#2 / #1}}

\newcommand{\mkwd}[1]{\mathsf{#1}}
\newcommand{\tkwd}[1]{\textsf{#1}}
\newcommand{\link}[2]{#1 \leftrightarrow #2}
\newcommand{\cut}[4]{\nu#1 \: (#3 \mid #4)}
\newcommand{\replicate}[2]{{!#1(#2)}}
\newcommand{\derelict}[2]{{?#1[#2]}}
\newcommand{\rec}[1]{\mkwd{rec}\:#1}
\newcommand{\corec}[4]{\mkwd{corec}\:{#1} \langle #2 \rangle (#3,#4)}
\newcommand{\corecf}[5]{\mkwd{corec}^{#1} #2 \langle #3 \rangle (#4,#5)}
\newcommand{\clabel}[1]{\mathit{#1}}
\renewcommand{\case}[2]{\mkwd{case}\:#1\:\{#2\}}
\newcommand{\sel}[2]{#1[\clabel{#2}]}
\newcommand{\gvsend}[1]{\gvconst{send} \app #1}
\newcommand{\gvreceive}[1]{\gvconst{receive} \app #1}
\newcommand{\gvlet}[3]{\mkwd{let}\;#1 = #2\;\mkwd{in}\;#3}
\newcommand{\gvletunit}[2]{#1; #2}
\newcommand{\gvconst}[1]{\ensuremath{\mathsf{#1}}}
\newcommand{\app}{\,}
\newcommand{\gvselect}[2]{\mkwd{select}\:#1\:#2}
\newcommand{\gvcase}[2]{\mkwd{case}\:#1\:\{#2\}}
\newcommand{\gvlink}[2]{\gvconst{link}\:(#1, #2)}
\newcommand{\gvfork}[2]{\gvconst{fork}\:(\lambda #1.#2)}
\newcommand{\gvwait}[1]{\gvconst{wait}\:#1}
%% \newcommand{\lrkwd}{\mkwd{cofix}}
%% \newcommand{\gvfix}[3]{\lrkwd\:#1\:#2 = #3}
%% \newcommand{\gvletrec}[4]{\gvlet{#1}{\gvfix{#1}{#2}{#3}}{#4}}
\newcommand{\gvserve}[2]{\gvconst{replicate} \app (\lambda #1.#2)}
\newcommand{\gvrequest}[1]{\gvconst{request} \app #1}

\newcommand{\gvifthen}[3]{\mkwd{if}\:#1\:\mkwd{then}\:#2\:\mkwd{else}\:#3}
\newcommand{\gvreceivek}[4]{\gvlet{({#1}, {#2})}{\gvreceive{#3}}{#4}}

\newcommand{\sd}[1]{\langle#1\rangle}

\newcommand{\clos}[3]{\lambda^{#1}\!#2.#3}

\newcommand{\true}{\mathrm{true}}
\newcommand{\false}{\mathrm{false}}

\newcommand{\key}{\mkwd}
\newcommand{\cofix}{\lrkwd}

\newcommand{\tocpbig}[1]{\left\llbracket #1 \right\rrbracket}
\newcommand{\togvbig}[1]{\stretchleftright{\llparenthesis}{#1}{\rrparenthesis}}

\newcommand{\tocpflip}[1]{\llceil{#1}\rrceil}

\newcommand{\tocp}[1]{\llbracket #1 \rrbracket}
\newcommand{\togv}[1]{\llparenthesis #1 \rrparenthesis}

% CP to GV configuration
%\newcommand{\togvc}[1]{\mathscr{C}\hspace{-.1ex}\togv{#1}}
\newcommand{\togvc}[1]{\mathcal{C}\togv{#1}}

\newcommand{\topi}[1]{({#1})^\star}

\newcommand{\un}{un}
\newcommand{\lin}{lin}

\newcommand{\mucp}{$\mu\mathrm{CP}$\xspace}
\newcommand{\mugv}{$\mu\mathrm{GV}$\xspace}
\newcommand{\gvpi}{$\mu\mathrm{GV}\pi$\xspace}

\newcommand{\ba}{\begin{array}}
\newcommand{\ea}{\end{array}}

\newcommand{\bl}{\ba[t]{@{}l@{}}}
\newcommand{\el}{\ea}

\newenvironment{code}{\small\[\bl}{\el\]\normalsize}

\newenvironment{equations}{\[\ba{@{}r@{~}c@{~}l@{}}}{\ea\]}
\newenvironment{eqs}{\ba{@{}r@{~}c@{~}l@{}}}{\ea}

\newenvironment{syntax}%
{\[\begin{array}{@{}lr@{\hspace{2px}}c@{\hspace{2px}}l}\ignorespaces}
{\end{array}\]\ignorespacesafterend}

\newcommand{\mcl}[2]{\multicolumn{#1}{l}{#2}}

\newcommand{\secref}[1]{(\S\ref{sec:#1})}
\newcommand{\lemref}[1]{Lemma~\ref{thm:#1}}
\newcommand{\thmref}[1]{Theorem~\ref{thm:#1}}
\newcommand{\figref}[1]{Figure~\ref{fig:#1}}

\newcommand{\todo}[1]{{\noindent\small\color{red} \framebox{\parbox{\dimexpr\linewidth-2\fboxsep-2\fboxrule}{\textbf{TODO:} #1}}}}
\newcommand{\nqed}{\ensuremath{\centernot\qed}}


\title{Joke or Something}
\author{J. Garrett Morris \and Sam Lindley \and Philip Wadler}
\institute{The University of Edinburgh \\
           \email{\{Garrett.Morris,Sam.Lindley,Philip.Wadler\}@ed.ac.uk} \vspace{-5mm}}

\begin{document}

\maketitle

\begin{abstract}

\end{abstract}

\section{I'm Pickin' Up Good Vibrations}

I'm intentionally ignoring replication for the time being.  I'm intentionally ignoring CP
polymorphism forever.

\subsection{Syntax}
\newcommand{\one}{\mathbf{1}}
\newcommand{\zero}{\mathbf{0}}
\begin{syntax}
  \text{Session types} & S & ::= & \gvout{T}{S} \mid \gvin{T}{S} \mid % \oplus \{ \overline{l:S} \} \mid \with \{ \overline{l:S} \} \mid
                                   \interm \mid \outterm \\
  \text{Types} & T,U & ::= & S \mid T \times U \mid \one \mid T + U \mid \zero \mid T \lto U \\ % \mid {!T} \\
  \text{Terms} & M,N & ::= & x \mid K \mid \lambda x:T.M \mid M\,N \\
  & & \mid & (M,N) \mid \gvlet{(x,y)}{M}{N} \\
  & & \mid & \mkwd{inl}\,M \mid \mkwd{inr}\,M \mid \gvcase{M}{\mkwd{inl}\,x \mapsto N; \mkwd{inr}\,x \mapsto N} \\
  & & \mid & () \mid \gvlet{()}{M}{N} \mid \mathsf{absurd}\,x \\
  \text{Constants} & K & ::= & \mkwd{send} \mid \mkwd{receive} \mid \mkwd{fork} \mid \mkwd{link}\\
\end{syntax}

\begin{mathpar}
\gvdual{\gvout{T}{S}} = \gvin{T}{\gvdual S}

\gvdual{\gvin{T}{S}} = \gvout{T}{\gvdual S}

\gvdual{\interm} = \outterm

\gvdual{\outterm} = \interm
\end{mathpar}

\subsection{Typing}

\begin{mathpar}
\inferrule
  { }
  {\{x:T\} \vdash x:T}

\inferrule
  {\Gamma, x:\interm, y:\interm \vdash M: T}
  {\Gamma, x:\interm \vdash M\{x/y\}:T}

\inferrule
  {\Gamma \vdash M: T}
  {\Gamma, x:\interm \vdash M:T}

\inferrule
  {\Gamma,x:T \vdash M:U}
  {\Gamma \vdash \lambda x:T.M : T \lto U}

\inferrule
  {\Gamma \vdash M: T \lto U \\
   \Delta \vdash N: T}
  {\Gamma,\Delta \vdash M\,N: U}

\inferrule
  {\Gamma \vdash M: T \\
   \Delta \vdash N: U}
  {\Gamma,\Delta \vdash (M,N): T \times U}

\inferrule
  {\Gamma \vdash M: T \times T' \\
   \Delta,x:T,y:T' \vdash N: U}
  {\Gamma,\Delta \vdash \gvlet{(x,y)}{M}{N} : U}

\inferrule
  {\Gamma \vdash M: T}
  {\Gamma \vdash \mkwd{inl}\,M: T+U}

\inferrule
  {\Gamma \vdash M: T+T' \\
   \Delta,x:T \vdash N: U \\
   \Delta,x:T' \vdash N': U}
  {\Gamma,\Delta \vdash \gvcase{M}{\mkwd{inl}\,x \mapsto N; \mkwd{inr}\,x \mapsto N'} : U}

\inferrule
  { }
  {\vdash (): \one}

\inferrule
  {\Gamma \vdash M: \one \\
   \Delta \vdash N: T}
  {\Gamma,\Delta \vdash \gvlet{()}{M}{N}: T}

\inferrule
  { }
  {\Gamma,x : \zero \vdash \mathsf{absurd}\,x : T}

\mkwd{send} : T \times \gvout{T}{S} \lto S

\mkwd{receive} : \gvin{T}{S} \lto T \times S
\\
\mkwd{fork} : (S \lto \outterm) \lto \gvdual{S}

\mkwd{link} : (S, \gvdual{S}) \lto \gvend
\end{mathpar}

\subsection{Evaluation}

\newcommand{\distinguish}[1]{\langle #1 \rangle}

\begin{syntax}
%  \text{Names} & \alpha,\beta \\
  \text{Values} & V & ::= & x \mid K \mid \lambda x:T.M \\ % \mid K\,V \\
  & & \mid & () \mid (V,V') \mid \mkwd{inl}\,V \mid \mkwd{inr}\,V  \\
  \text{Expression contexts} & E & ::= &  [] \mid E\,M \mid V\,E \mid \gvlet{()}{E}{M} \\
  & & \mid & (E,M) \mid (V,E) \mid \gvlet{(x,y)}{E}{M} \\
  & & \mid & \mkwd{inl}\,E \mid \mkwd{inr}\,E \mid \gvcase{E}{\mkwd{inl}\,x \mapsto N; \mkwd{inr}\,x \mapsto N'} \\
  \text{Configurations} & P & ::= & M \mid \distinguish M \mid P \parallel P' \mid (\nu x) P \\
  \text{Configuration contexts} & C & ::= & E \mid \distinguish E \mid C \parallel P \mid (\nu x)C
\end{syntax}
Configuration typing:
\begin{mathpar}
\inferrule
  {\Gamma \vdash M: T}
  {\Gamma \vdash \distinguish M}

\inferrule
  {\Gamma \vdash M: \outterm}
  {\Gamma \vdash M}

\inferrule
  {\Gamma \vdash P \\
   \Gamma' \vdash P'}
  {\Gamma,\Gamma' \vdash P \parallel P'}

\inferrule
  {\Gamma, x:S, x:\gvdual{S} \vdash P}
  {\Gamma \vdash (\nu x) P}
\end{mathpar}
Wait, $\Gamma$ is a multiset?  When did that happen? Term reductions:
\begin{align*}
  (\lambda x:T. M)\,V &\teval M\{V/x\} \\
  \gvlet{()}{()}{M} &\teval M \\
  \gvlet{(x,y)}{(V,V')}{M} &\teval M\{V/x,V'/y\} \\
  \gvcase{(\mkwd{inl}\,V)}{\mkwd{inl}\,x \mapsto N; \mkwd{inr}\,x \mapsto N'} &\teval N\{V/x\} \\
  E[M] &\teval E[M'] \qquad \text{if $M \teval M'$}
\end{align*}
Configuration (context) equivalence:
\begin{mathpar}
C_0 \parallel C_1 \equiv C_1 \parallel C_0

C_0 \parallel (C_1 \parallel C_2) \equiv (C_0 \parallel C_1) \parallel C_2

C_0 \parallel (\nu x)C_1 \equiv (\nu x)(C_0 \parallel C_1) \text{ if $x \not\in \mathsf{fv}(C_0)$}
\end{mathpar}
Configuration reduction:
\begin{mathpar}
\inferrule
  { }
  {C[\mkwd{send}\,(V,x)] \parallel C'[\mkwd{receive}\,x] \ceval C[x] \parallel C'[(V,x)]}

\inferrule
  {\text{$x$ fresh}}
  {C[\mkwd{fork}\,V] \ceval (\nu x)(C[x] \parallel (V\,x))}

\inferrule
  {M \teval M'}
  {M \ceval M'}

\inferrule
  {M \teval M'}
  {\distinguish M \ceval \distinguish {M'}}

\inferrule
  {C_0 \ceval C_0'}
  {C_0 \parallel C_1 \ceval C_0' \parallel C_1}

\inferrule
  {C_0 \equiv C_0' \\
   C_0' \ceval C_1' \\
   C_1' \equiv C_1}
  {C_0 \ceval C_1}

\inferrule
  {C \ceval C'}
  {(\nu x)C \ceval (\nu x)C'}
\end{mathpar}
Some things that might be useful
\begin{theorem}
If $\Gamma \vdash P$ and $P \ceval P'$ then $\Gamma \vdash P'$.
\end{theorem}
\begin{theorem}
If $\Gamma \vdash M: T$, then for all $P$ such that $\distinguish M \ceval^\star P
\centernot\ceval$, there is a value $V$ such that $P \equiv (\nu x_1,\dots,x_n)(\distinguish V
\parallel P')$.
\end{theorem}

\subsection{Deadlock and Its Absence}

Blocked terms:
\[
  \mathsf{blocked}(x,P) \stackrel{\mathrm{def}}{\iff} P \equiv C[\mathsf{receive}\,x] \lor \exists M. P \equiv C[\mathsf{send}\,(M,x)]
\]

Now we can say something about deadlock.  A process $P$ with free variables $y_1,\dots,y_l$ is deadlock-free if, for all $Q$ such that
$P \ceval^{\star} Q \centernot\ceval$, $Q \equiv (\nu x_1,\dots,x_m)(\distinguish V \parallel E_1 \parallel \dots
\parallel E_n)$ and, for each $1 \leq i \leq m, 1 \leq j \leq n$, if $\mathsf{blocked}(x_i,E_j)$ then $x_i \in
\mathsf{fv}(V)$ or $x_i \in \{ y_1, \dots, y_l \}$.

\begin{theorem}
If $\Gamma \vdash M: T$, then $\distinguish M$ is deadlock-free.
\end{theorem}

\section{CP Bollocks}

Suppose that we write $\tocp{-}$ for GV to CP translation, and $\togv{-}$ for CP to GV translation.
Note that $\tocp{P}z$ will only be defined when $\Gamma \vdash P$, and when $P$ has a single
distinguished expression.  I suspect that the following will be useful at some point.
\begin{lemma}
  If $\Gamma \vdash M: T$ and $\distinguish M \ceval^\star P$, then there are $x_1,\dots,x_n$, $M$ and $P'$
  such that $P \equiv (\nu x_1,\dots,x_n)(\distinguish M \parallel P')$.
\end{lemma}

We might expect results something like the following.

\begin{theorem}
  If $\Gamma \vdash P$, where $P = (\nu x_1,\dots,x_m)(\distinguish M \parallel P')$ and $\Gamma
  \vdash M: T$, then $\tocp{P}z \vdash \tocp{\Gamma}, z:\tocp{T}$.
\end{theorem}

\begin{theorem}
  If $\Gamma \vdash P$ and $P \ceval P'$, then $\tocp{P}z \longrightarrow Q$ such that $\togv{Q}
  \equiv P'$.  What to do with $z$?!
\end{theorem}

\end{document}
