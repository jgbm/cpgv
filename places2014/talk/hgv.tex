%\documentclass{beamer}
%
\documentclass[serif]{beamer}

\usepackage[T1]{fontenc}
\usepackage{fourier}

%\usepackage{beamerthemesplit}


\setbeamersize{text margin left = 2em}

\usepackage[all]{xy}
\usepackage{amssymb}
\usepackage{amsmath}
\usepackage{stmaryrd}
\usepackage{xspace}
\usepackage{url}
\usepackage{hyperref}

\usetheme{Frankfurt}

% disable useless navigation symbols
\setbeamertemplate{navigation symbols}{}

% disable the section information
\setbeamertemplate{headline}{}

% use triangles for bullets
\setbeamertemplate{itemize items}[triangle]

\title{Sessions as propositions}
%\subtitle{}
\author{Sam Lindley}
%\email{Sam.Lindley@ed.ac.uk}
\institute{Laboratory for Foundations of Computer Science \\
           The University of Edinburgh \\[.2cm]
           {\texttt\tiny Sam.Lindley@ed.ac.uk}}
\date{12th April 2014}


%% macros


\begin{document}

\begin{frame}
\titlepage

\begin{center}
(joint work with Garrett Morris)
\end{center}


\end{frame}


%% \begin{frame}
%% \frametitle{Background}


%% \end{frame}

\newcommand{\hgv}{HGV\xspace}
\newcommand{\hgvpi}{HGV$\pi$\xspace}

\begin{frame}[fragile]
\frametitle{Propositions as sessions (intuitionistically)}

Caires and Pfenning (2010): Curry-Howard correspondence between dual
intuitionistic linear logic and session types
\begin{center}
\begin{tabular}{l@{\quad$\Longleftrightarrow$\quad}l}
  propositions & session types \\
  proofs & processes \\
  cut elimination & communication \\
\end{tabular}
\end{center}

\begin{description}
\item[$\pi$DILL] a process calculus for dual intuitionistic linear
  logic
\end{description}

\end{frame}

\begin{frame}[fragile]
\frametitle{Propositions as sessions (classically)}
Wadler (2012): Curry-Howard correspondence between session types and
classical linear logic
\begin{itemize}
\item CP: a process calculus for classical linear logic
\item GV: a functional programming language with session types (based
  on Gay and Vasconcelos, 2010)
\item Mapping from GV to CP
\item {\color{red} {No translation from CP to GV}}
\end{itemize}

\begin{description}
\item[CP] Caires and Pfenning / Classical Processes
\item[GV] Gay and Vasconcelos / Good Variation
\end{description}
\end{frame}

\begin{frame}[fragile]
\frametitle{This work}

\begin{description}
\item[\hgv] Harmonious GV
\end{description}

\begin{itemize}
\item \hgv = GV + missing features
\item \hgvpi is the session-typed fragment of \hgv
\item Translations
\[
\hspace{-2cm}
\xymatrix{%
  \mbox{\hgv}
  \ar@<+.7ex>[d]
\\
\raisebox{0ex}[2ex]{\hgvpi}
  \ar@<+.7ex>@{^{(}->}[u]
  \ar@<+.7ex>[r]
&
  \mbox{CP}
  \ar@<+.7ex>[l]
}
\]
%% \begin{itemize}
%% \item  \hgvpi $\hookrightarrow$ \hgv
%% \item  \hgv $\longrightarrow$ \hgvpi
%% \item  \hgvpi $\longrightarrow$ CP
%% \item  CP $\longrightarrow$ \hgvpi
%% \end{itemize}
\item CP, \hgv, and \hgvpi are all equally expressive
\end{itemize}

%% We have prototype implementations of CP and \hgv + various
%% extensions. Implementations use big-step semantics.
\end{frame}

\end{document}
